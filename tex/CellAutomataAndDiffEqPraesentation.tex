% !TEX TS-program = pdflatex
% !TEX encoding = UTF-8 Unicode

% This file is a template using the "beamer" package to create slides for a talk or presentation
% - Giving a talk on some subject.
% - The talk is between 15min and 45min long.
% - Style is ornate.

% MODIFIED by Jonathan Kew, 2008-07-06
% The header comments and encoding in this file were modified for inclusion with TeXworks.
% The content is otherwise unchanged from the original distributed with the beamer package.

\documentclass{beamer}


% Copyright 2004 by Till Tantau <tantau@users.sourceforge.net>.
%
% In principle, this file can be redistributed and/or modified under
% the terms of the GNU Public License, version 2.
%
% However, this file is supposed to be a template to be modified
% for your own needs. For this reason, if you use this file as a
% template and not specifically distribute it as part of a another
% package/program, I grant the extra permission to freely copy and
% modify this file as you see fit and even to delete this copyright
% notice. 


\mode<presentation>
{
  \usetheme{Warsaw}
  % or ...

  \setbeamercovered{transparent}
  % or whatever (possibly just delete it)
}


\usepackage[english]{babel}
% or whatever

\usepackage[utf8]{inputenc}
% or whatever

\usepackage{times}
\usepackage[T1]{fontenc}
\usepackage{listings}

\lstdefinelanguage{JavaScript}{
  keywords={typeof, new, true, false, catch, function, return, null, catch, switch, var, if, in, while, do, else, case, break},
  keywordstyle=\color{blue}\bfseries,
  ndkeywords={class, export, boolean, throw, implements, import, this},
  ndkeywordstyle=\color{darkgray}\bfseries,
  identifierstyle=\color{black},
  sensitive=false,
  comment=[l]{//},
  morecomment=[s]{/*}{*/},
  commentstyle=\color{purple}\ttfamily,
  stringstyle=\color{red}\ttfamily,
  morestring=[b]',
  morestring=[b]"
}

\lstset{
   language=JavaScript,
   backgroundcolor=\color{lightgray},
   extendedchars=true,
   basicstyle=\footnotesize\ttfamily,
   showstringspaces=false,
   showspaces=false,
   %numbers=left,
   numberstyle=\footnotesize,
   numbersep=9pt,
   tabsize=2,
   breaklines=true,
   showtabs=false,
   captionpos=b
}


% Or whatever. Note that the encoding and the font should match. If T1
% does not look nice, try deleting the line with the fontenc.


\title % (optional, use only with long paper titles)
{Zelluläre Automaten und Differentialgleichungen}

\author[D.Z., F.L.] % (optional, use only with lots of authors)
{Detlev Ziereisen \and Florian Lüthi}
% - Use the \inst{?} command only if the authors have different
%   affiliation.

\institute[ZHAW] % (optional, but mostly needed)
{
  ZHAW (HSZ-T)
}
% - Use the \inst command only if there are several affiliations.
% - Keep it simple, no one is interested in your street address.

\date[Short Occasion] % (optional)
{16.06.2012}



% If you have a file called "university-logo-filename.xxx", where xxx
% is a graphic format that can be processed by latex or pdflatex,
% resp., then you can add a logo as follows:

% \pgfdeclareimage[height=0.5cm]{university-logo}{university-logo-filename}
% \logo{\pgfuseimage{university-logo}}



% Delete this, if you do not want the table of contents to pop up at
% the beginning of each subsection:
\AtBeginSubsection[]
{
  \begin{frame}<beamer>{Outline}
    \tableofcontents[currentsection,currentsubsection]
  \end{frame}
}


% If you wish to uncover everything in a step-wise fashion, uncomment
% the following command: 

%\beamerdefaultoverlayspecification{<+->}

\def\signed #1{{\leavevmode\unskip\nobreak\hfil\penalty50\hskip2em
  \hbox{}\nobreak\hfil(#1)%
  \parfillskip=0pt \finalhyphendemerits=0 \endgraf}}

\newsavebox\mybox
\newenvironment{aquote}[1]
  {\savebox\mybox{#1}\begin{quote}}
  {\signed{\usebox\mybox}\end{quote}}


\begin{document}

\begin{frame}
  \titlepage
\end{frame}

\begin{frame}
\begin{aquote}{Konrad Zuse, Rechnender Raum, 1969}
Das gesamte Geschehen im Universum ist das Ergebnis der Arbeit eines gigantischen Zellulären Automaten.
\end{aquote}
\end{frame}

\begin{frame}{Agenda}
  \tableofcontents
  % You might wish to add the option [pausesections]
\end{frame}


% Since this a solution template for a generic talk, very little can
% be said about how it should be structured. However, the talk length
% of between 15min and 45min and the theme suggest that you stick to
% the following rules:  

% - Exactly two or three sections (other than the summary).
% - At *most* three subsections per section.
% - Talk about 30s to 2min per frame. So there should be between about
%   15 and 30 frames, all told.

\section{Ziel des Projekts}

\begin{frame}{Ziel des Projekts}{}
  % - A title should summarize the slide in an understandable fashion
  %   for anyone how does not follow everything on the slide itself.

Zellulärer Automat, der...

\begin{itemize}
	\item allgemein ist,
	\item Differentialgleichungen lösen kann,
	\item cool aussieht,
	\item portabel ist.
\end{itemize}

\end{frame}


\section{Theorie}

\subsection[Zell]{Zelluläre Automaten}

\begin{frame}{Definition}
  % - A title should summarize the slide in an understandable fashion
  %   for anyone how does not follow everything on the slide itself.
Ein Zellulärer Automat hat:
  \begin{itemize}
\item einen Zellularraum $R$,
\item eine endliche Nachbarschaft $N$, wobei $\forall r \in R \left(N_r \subset R\right)$,
\item eine Zustandsmenge $Q$,
\item eine Überführungsfunktion $\delta: Q^{|N| + 1}\mapsto Q$.
	
  \end{itemize}
\end{frame}

\begin{frame}{Wolfram's eindimensionales Universum}
  % - A title should summarize the slide in an understandable fashion
  %   for anyone how does not follow everything on the slide itself.
  \begin{itemize}
\item eindimensional
\item $|N| = 2$, $|Q| = 2$
\item $\Rightarrow |\mathrm{img}(\delta)| = 256$
\item $\Rightarrow$ 256 verschiedene Automaten
\item Automat Nr. 110 ist turing-complete!
	
  \end{itemize}
\end{frame}

\subsection{Differentialgleichungen}

\begin{frame}{Differentiation}
\begin{eqnarray*}
	\left(\frac {\Delta}{\Delta \vec x} u\right)_{\vec x} &=& u_{\vec x} - \sum\limits_{i=1}^{\dim(\vec x)} u_{\vec x - \vec e_i} \\
\left( \frac{\Delta^2}{\Delta \vec x^2} u\right)_{\vec x} &=& u_{\vec x} - 2\cdot\sum\limits_{i=1}^{\dim(\vec x)} u_{\vec x - \vec e_i} + \sum\limits_{i=1}^{\dim(\vec x)} u_{\vec x - 2\vec e_i}
\end{eqnarray*}
\end{frame}

\begin{frame}{Integration}
\[
\begin{bmatrix}
\begin{array}{c|c}
a & B\\
\hline     & c \\
\end{array}
\end{bmatrix} = \begin{bmatrix}
\begin{array}{c|cccc}
  \alpha_1 & 0  \\
  \alpha_2    & \beta_{2,1} & 0\\
  \vdots & \vdots & \vdots& \ddots\\
  \alpha_m    & \beta_{m,1} & \beta_{m,2}& \cdots & 0\\
  \hline & \gamma_1    & \gamma_2   & \cdots & \gamma_m\\
\end{array}

\end{bmatrix}
\]
\end{frame}

\subsection{Beides zusammen}

\begin{frame}{Beides zusammen}
\begin{itemize}
\item $R$ beinhaltet die diskreten Werte der zu findenden "Funktion"
\item $\Delta \vec{x}$ ist grob diskretisiert
\item $\Delta t$ ist fein diskretisiert
\item $(Q, D) \in R$ sind die Zustandsinformationen der Zellen (Differentiale nach der Zeit)
\item In der Übergangsfunktion $\delta$ ist die Differentialgleichung.
\end{itemize}

\end{frame}

\begin{frame}{Die Welle}
\[
\frac {\partial^2 u} {\partial \vec x^2} = k \cdot \frac {\partial^2 u}{\partial t^2}
\]
\end{frame}

\begin{frame}{Die Welle}
\begin{align*}
u_i &&&  u_i(t) +  \frac{\partial u_i(t+\Delta t)}{\partial t}\Delta t \\
\downarrow &&& \uparrow \\
\frac{\partial u_i}{\partial \vec x} = u_i - \sum u_{i-1} &&& \frac{\partial u_i(t+\Delta t)}{\partial t} = \frac{\partial u_i(t)}{\partial t} + \frac{\partial^2 u}{\partial t^2}\Delta t \\
\downarrow &&& \uparrow \\
  -2u_i + \sum u_{i-1} + \sum u_{i+1} & \underbrace{\longrightarrow}_{\frac{\partial^2 u}{\partial t^2} = k\cdot \frac{\partial^2 u}{\partial \vec{x}^2}} && \frac{\partial^2 u}{\partial t^2}
\end{align*}
\end{frame}

\section{Implementation}

\subsection{Testing}

\begin{frame}[fragile]{Testing mit Node.js und Mocha}
\begin{lstlisting}
if (typeof(module) != "undefined") {
  module.exports = DiffusionEquation;
  var ATusk = require('../src/ATusk.js');
  var VNA = require('../../src/VonNeumannNeighbourhood.js');
}
function DiffusionEquation() {
  /* ... */
}

\end{lstlisting}
\end{frame}

\begin{frame}[fragile]{Testing mit Node.js und Mocha}
\begin{lstlisting}
describe('GameOfLife', function() {
  it('should comply to rule 1', function() {
    var cell = {x:1, y:1, currentData: {status: 0}};
			
    var model = [
      [creator(1), creator(1), creator(1)],
      [creator(0), cell      , creator(0)],
      [creator(0), creator(0), creator(0)]
    ];
		
    assert(game, cell, model, 1);
				
  });
}
\end{lstlisting}
\end{frame}

\subsection{Strategy}

\begin{frame}[fragile]{Strategy}
\begin{lstlisting}
ViewUtils.bindStrategiesToCombobox(null, this.doc.painter, PainterFactory.types, function(p) {
  return p.name;
}, (function(view) { return function(p) {
  view.primaryPainter = PainterFactory.create(p, view.doc.primaryCanvas, view);
  view.secondaryPainter = PainterFactory.create(p, view.doc.secondaryCanvas, view);
  view.updatePainterScaling();
  view.doc.viscositySelector.onchange();
  if (view.automata.tusk != null) {
    view.primaryPainter.pool = view.automata.tusk.primaryPool;
  }
};})(this));	
\end{lstlisting}
\end{frame}

\subsection{Painting}

\begin{frame}[fragile]{Painting: Vektoriell}
\begin{lstlisting}
this.paintCell = function(cell) {
		
  var x = cell.x * this.scaling.x;
  var y = cell.y * this.scaling.y;
  var baseColor = this.baseColor;

  var color = ViewUtils.getFormattedColor(this.pool.getValue(cell), baseColor.r, baseColor.g, baseColor.b);
  this.context.fillStyle = color;
  this.context.fillRect(x, y, this.scaling.x, this.scaling.y);
};
\end{lstlisting}
\end{frame}

\begin{frame}[fragile]{Painting: Pixelesk}
\begin{lstlisting}
this.begin = function() {
  this.imageData = this.context.createImageData(this.view.CANVAS_WIDTH, this.view.CANVAS_HEIGHT);
  this.buf = new ArrayBuffer(this.imageData.data.length);
  this.buf8 = new Uint8ClampedArray(this.buf);
  this.data = new Uint32Array(this.buf);
};
	
this.end = function() {
  this.imageData.data.set(this.buf8);
  this.context.putImageData(this.imageData, 0, 0);
};
\end{lstlisting}
\end{frame}
\begin{frame}[fragile]
\begin{lstlisting}
this.paintCell = function(cell) {
  var x = cell.x * this.scaling.x;
  var y = cell.y * this.scaling.y;		
  var baseColor = this.baseColor;
  var color = ViewUtils.getColor(this.pool.getValue(cell), baseColor.r, baseColor.g, baseColor.b);
  for (var ix = x; ix < x + this.scaling.x; ix++) {
    for (var iy = y; iy < y + this.scaling.y; iy++) {
      var p = (iy * this.view.CANVAS_WIDTH + ix);			
      this.data[p] = (255   << 24) |
            		(color.b << 16) |
            		(color.g <<  8) |
             		color.r;
    }
  }
};
\end{lstlisting}
\end{frame}

\subsection{Lösung der Gleichung}

\begin{frame}[fragile]{Wo ist $\delta$?}
\begin{lstlisting}
this.calcDifferentials = function(cell, dt, get) {
  var c = 1.0 / this.viscosity;
  var u = get(cell).u;		
  var udx = u - (get(cell.neighbours.w).u + get(cell.neighbours.n).u) / 2;
  var udxdx = -2 * u + (get(cell.neighbours.w).u + get(cell.neighbours.n).u) / 2
    + (get(cell.neighbours.e).u + get(cell.neighbours.s).u) / 2;	
  var udtdt = udxdx * c;
  var udt = get(cell).udt + udtdt * dt;
		
  return {
    udt: udt, udtdt: udtdt,			
    udx: udx, udxdx: udxdx
  };
}
\end{lstlisting}
\end{frame}

\begin{frame}[fragile]{Integration}
\begin{lstlisting}
  this.integrate = function(automata) {
    var dt = 1 / automata.tusk.slices;
    for (var t = 0; t < automata.tusk.slices; t++) {		
      automata.forEachCell(function(cell) {
        var differentials = automata.tusk.calcDifferentials(cell, dt, function(cell2) {
          return cell2.currentData;
        });
        cell.nextData = automata.tusk.applyDifferentials(cell, dt, differentials, function(cell2) {
          return cell2.currentData;
        })
      });
      automata.forEachCell(function(cell) {
        cell.currentData = cell.nextData;
      });
  }
}
\end{lstlisting}
\end{frame}

\section{Fazit}

\begin{frame}{Fazit}
\begin{itemize}
\item Zuse ist wohl auf dem richtigen Weg
\item Portabilität nur beschränkt möglich
\item Erweiterungspotenzial vorhanden
\end{itemize}
\end{frame}

\begin{frame}{Fazit}
\begin{itemize}
\item Fragen?
\item Verbindlichsten Dank für die Aufmerksamkeit
\end{itemize}
\end{frame}

\end{document}


