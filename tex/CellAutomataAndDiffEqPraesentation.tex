% !TEX TS-program = pdflatex
% !TEX encoding = UTF-8 Unicode

% This file is a template using the "beamer" package to create slides for a talk or presentation
% - Giving a talk on some subject.
% - The talk is between 15min and 45min long.
% - Style is ornate.

% MODIFIED by Jonathan Kew, 2008-07-06
% The header comments and encoding in this file were modified for inclusion with TeXworks.
% The content is otherwise unchanged from the original distributed with the beamer package.

\documentclass{beamer}


% Copyright 2004 by Till Tantau <tantau@users.sourceforge.net>.
%
% In principle, this file can be redistributed and/or modified under
% the terms of the GNU Public License, version 2.
%
% However, this file is supposed to be a template to be modified
% for your own needs. For this reason, if you use this file as a
% template and not specifically distribute it as part of a another
% package/program, I grant the extra permission to freely copy and
% modify this file as you see fit and even to delete this copyright
% notice. 


\mode<presentation>
{
  \usetheme{Warsaw}
  % or ...

  \setbeamercovered{transparent}
  % or whatever (possibly just delete it)
}


\usepackage[english]{babel}
% or whatever

\usepackage[utf8]{inputenc}
% or whatever

\usepackage{times}
\usepackage[T1]{fontenc}
\usepackage{listings}

\lstdefinelanguage{JavaScript}{
  keywords={typeof, new, true, false, catch, function, return, null, catch, switch, var, if, in, while, do, else, case, break},
  keywordstyle=\color{blue}\bfseries,
  ndkeywords={class, export, boolean, throw, implements, import, this},
  ndkeywordstyle=\color{darkgray}\bfseries,
  identifierstyle=\color{black},
  sensitive=false,
  comment=[l]{//},
  morecomment=[s]{/*}{*/},
  commentstyle=\color{purple}\ttfamily,
  stringstyle=\color{red}\ttfamily,
  morestring=[b]',
  morestring=[b]"
}

\lstset{
   language=JavaScript,
   backgroundcolor=\color{lightgray},
   extendedchars=true,
   basicstyle=\footnotesize\ttfamily,
   showstringspaces=false,
   showspaces=false,
   %numbers=left,
   numberstyle=\footnotesize,
   numbersep=9pt,
   tabsize=2,
   breaklines=true,
   showtabs=false,
   captionpos=b
}


% Or whatever. Note that the encoding and the font should match. If T1
% does not look nice, try deleting the line with the fontenc.


\title[Short Paper Title] % (optional, use only with long paper titles)
{Zelluläre Automaten und Differentialgleichungen}

\subtitle
{Yeah.} % (optional)

\author[Author, Another] % (optional, use only with lots of authors)
{Detlev Ziereisen\inst{1} \and Florian Lüthi\inst{1}}
% - Use the \inst{?} command only if the authors have different
%   affiliation.

\institute[ZHAW] % (optional, but mostly needed)
{
  \inst{1}%
  Department of Computer Science\\
  ZHAW
}
% - Use the \inst command only if there are several affiliations.
% - Keep it simple, no one is interested in your street address.

\date[Short Occasion] % (optional)
{16.06.2012}

\subject{Talks}
% This is only inserted into the PDF information catalog. Can be left
% out. 



% If you have a file called "university-logo-filename.xxx", where xxx
% is a graphic format that can be processed by latex or pdflatex,
% resp., then you can add a logo as follows:

% \pgfdeclareimage[height=0.5cm]{university-logo}{university-logo-filename}
% \logo{\pgfuseimage{university-logo}}



% Delete this, if you do not want the table of contents to pop up at
% the beginning of each subsection:
\AtBeginSubsection[]
{
  \begin{frame}<beamer>{Outline}
    \tableofcontents[currentsection,currentsubsection]
  \end{frame}
}


% If you wish to uncover everything in a step-wise fashion, uncomment
% the following command: 

%\beamerdefaultoverlayspecification{<+->}


\begin{document}

\begin{frame}
  \titlepage
\end{frame}

\begin{frame}{Outline}
  \tableofcontents
  % You might wish to add the option [pausesections]
\end{frame}


% Since this a solution template for a generic talk, very little can
% be said about how it should be structured. However, the talk length
% of between 15min and 45min and the theme suggest that you stick to
% the following rules:  

% - Exactly two or three sections (other than the summary).
% - At *most* three subsections per section.
% - Talk about 30s to 2min per frame. So there should be between about
%   15 and 30 frames, all told.

\section{Ziel}

\begin{frame}{Ziel}{}
  % - A title should summarize the slide in an understandable fashion
  %   for anyone how does not follow everything on the slide itself.

Zellulärer Automat, der...

\begin{itemize}
	\item allgemein ist,
	\item Differentialgleichungen lösen kann,
	\item cool aussieht,
	\item portabel ist.
\end{itemize}

\end{frame}


\section{Theorie}

\subsection[Zell]{Zelluläre Automaten}

\begin{frame}{Definition}
  % - A title should summarize the slide in an understandable fashion
  %   for anyone how does not follow everything on the slide itself.
Ein Zellulärer Automat hat:
  \begin{itemize}
\item einen Zellularraum $R$,
\item eine endliche Nachbarschaft $N$, wobei $\forall r \in R \left(N_r \subset R\right)$,
\item eine Zustandsmenge $Q$,
\item eine Überführungsfunktion $\delta: Q^{|N| + 1}\mapsto Q$.
	
  \end{itemize}
\end{frame}

\begin{frame}{Wolfram's eindimensionales Universum}
  % - A title should summarize the slide in an understandable fashion
  %   for anyone how does not follow everything on the slide itself.
Ein Zellulärer Automat hat:
  \begin{itemize}
\item eindimensional
\item $|N| = 2$, $|Q| = 2$
\item $\Rightarrow |\mathrm{img}(\delta)| = 256$
\item $\Rightarrow$ 256 verschiedene Automaten
\item Automat Nr. 110 ist turing-complete!
	
  \end{itemize}
\end{frame}

\subsection{Differentialgleichungen}

\begin{frame}{Differentiation}
\begin{eqnarray*}
	\left(\frac {\Delta}{\Delta \vec x} u\right)_{\vec x} &=& u_{\vec x} - \sum\limits_{i=1}^{\dim(\vec x)} u_{\vec x - \vec e_i} \\
\left( \frac{\Delta^2}{\Delta \vec x^2} u\right)_{\vec x} &=& u_{\vec x} - 2\cdot\sum\limits_{i=1}^{\dim(\vec x)} u_{\vec x - \vec e_i} + \sum\limits_{i=1}^{\dim(\vec x)} u_{\vec x - 2\vec e_i}
\end{eqnarray*}
\end{frame}

\begin{frame}{Integration}
\[
\begin{bmatrix}
\begin{array}{c|c}
a & B\\
\hline     & c \\
\end{array}
\end{bmatrix} = \begin{bmatrix}
\begin{array}{c|cccc}
  \alpha_1 & 0  \\
  \alpha_2    & \beta_{2,1} & 0\\
  \vdots & \vdots & \vdots& \ddots\\
  \alpha_m    & \beta_{m,1} & \beta_{m,2}& \cdots & 0\\
  \hline & \gamma_1    & \gamma_2   & \cdots & \gamma_m\\
\end{array}

\end{bmatrix}
\]
\end{frame}

\subsection{Beides zusammen}

\begin{frame}{Beides zusammen}
\begin{itemize}
\item $R \approx \vec{x}$ in einer, zwei oder drei Dimensionen =< Florian
\item  $(Q, D) \in R$ mit $Q$ als einer Menge von berechnungsfernen Zustandsinformationen und den Differentialen nach der Zeit
\[
D = \left( u, \frac{\partial u}{t}, \frac{\partial^2 u}{\partial t^2}, \cdots, \frac{\partial^n u}{t^n} \right)  \in \mathbb{R}^n
\]
\item Eine neue Generation entspricht jeweils der fortgelaufenen Zeit $\partial t$, welche sehr fein diskretisiert werden muss
\item In der Übergangsfunktion $\delta$ steckt die eigentliche Differentialgleichung. In der Regel verändert sie nur die Elemente von $D$. Sollte die Differentialgleichung Terme mit verschiedenen Ordnungen enthalten, wird die Gleichung unter Zuhilfenahme entsprechender Hilfsgleichungen $\lambda_1, \lambda_2, \dots, \lambda_n$ in ein äquivalentes System gewöhnlicher Differentialgleichungen umgeformt. 
\end{itemize}

\end{frame}

\begin{frame}{Die Welle}
\[
\frac {\partial^2 u} {\partial \vec x^2} = k \cdot \frac {\partial^2 u}{\partial t^2}
\]
\end{frame}

\begin{frame}{Die Welle}
\begin{align*}
u_i &&&  u_i(t) +  \frac{\partial u_i(t+\Delta t)}{\partial t}\Delta t \\
\downarrow &&& \uparrow \\
\frac{\partial u_i}{\partial \vec x} = u_i - \sum u_{i-1} &&& \frac{\partial u_i(t+\Delta t)}{\partial t} = \frac{\partial u_i(t)}{\partial t} + \frac{\partial^2 u}{\partial t^2}\Delta t \\
\downarrow &&& \uparrow \\
  -2u_i + \sum u_{i-1} + \sum u_{i+1} & \underbrace{\longrightarrow}_{\frac{\partial^2 u}{\partial t^2} = k\cdot \frac{\partial^2 u}{\partial \vec{x}^2}} && \frac{\partial^2 u}{\partial t^2}
\end{align*}
\end{frame}

\section{Implementation}

\subsection{JavaScript}

\begin{frame}
\begin{itemize}
\item Portabel
\item sdfsdfs
\item Closures
\end{itemize}
\end{frame}

\subsection{Testing}
\begin{lstlisting}
if (typeof(module) != "undefined") {
	module.exports = DiffusionEquation;
	var ATusk = require('../src/ATusk.js');
	var VNA = require('../../src/VonNeumannNeighbourhood.js');
}

function DiffusionEquation() {
  /* ... */
}
\end{lstlisting}
\subsection{Canvas}

\section{Résumé}

\begin{frame}{Make Titles Informative. Use Uppercase Letters.}{Subtitles are optional.}
  % - A title should summarize the slide in an understandable fashion
  %   for anyone how does not follow everything on the slide itself.

  \begin{itemize}
  \item
    Use \texttt{itemize} a lot.
  \item
    Use very short sentences or short phrases.
  \end{itemize}
\end{frame}

\begin{frame}{Make Titles Informative.}

  You can create overlays\dots
  \begin{itemize}
  \item using the \texttt{pause} command:
    \begin{itemize}
    \item
      First item.
      \pause
    \item    
      Second item.
    \end{itemize}
  \item
    using overlay specifications:
    \begin{itemize}
    \item<3->
      First item.
    \item<4->
      Second item.
    \end{itemize}
  \item
    using the general \texttt{uncover} command:
    \begin{itemize}
      \uncover<5->{\item
        First item.}
      \uncover<6->{\item
        Second item.}
    \end{itemize}
  \end{itemize}
\end{frame}


\subsection{Second Subsection}

\begin{frame}{Make Titles Informative.}
\end{frame}

\begin{frame}{Make Titles Informative.}
\end{frame}



\section*{Summary}

\begin{frame}{Summary}

  % Keep the summary *very short*.
  \begin{itemize}
  \item
    The \alert{first main message} of your talk in one or two lines.
  \item
    The \alert{second main message} of your talk in one or two lines.
  \item
    Perhaps a \alert{third message}, but not more than that.
  \end{itemize}
  
  % The following outlook is optional.
  \vskip0pt plus.5fill
  \begin{itemize}
  \item
    Outlook
    \begin{itemize}
    \item
      Something you haven't solved.
    \item
      Something else you haven't solved.
    \end{itemize}
  \end{itemize}
\end{frame}


\end{document}


